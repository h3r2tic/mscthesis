% Chapter 5

\chapter{ Nucled }
\label{Chapter5}
\lhead{Chapter 5. \emph{ Nucled }}

The entire machinery of Nucleus may be operated from source code and configuration files. While the \emph{ domain specific language } created for this purpose makes it possible and relatively easy, real-world usage scenarios demand a more intuitive solution. The obvious one for Nucleus was to use a graph-based visual design tool, similar to the ones used by [TODO: refs].

I have created such a tool, dubbed "Nucled". While it's not yet feature-complete, its capabilities include:

\begin{itemize}
\item kernel graph creation and manipulation
\item editing of the Cg source code of kernels,
\item real-time preview of graphical effects,
\item manipulation of 3D scenes,
\item application and creation of new materials,
\item manipulation of post-processing pipelines.
\end{itemize}

The interface of Nucled is based on a custom Graphical User Interface (GUI) toolkit [TODO: ref Hybrid], which is able to use the low-level layer of Nucleus as the display backend. This approach guarantees smooth integration of the rendering framework within Nucled and serves as a test of the system. While the editor looks like a regular desktop application, it's in fact rendered entirely by Nucleus, down from each button, text label, up to 3D scene viewports and material previews.

The machinery of material previews.

Some info about how post-processing previews are displayed.

Info about the issues mentioned by Whitted (mentioned in McGuire's paper) about type mismatches and how they're mostly non-existent in Nucled thanks to the semantic type system.

Mini-conclusion as how Nucled is a sketch but may become a complete solution for authoring effects.
